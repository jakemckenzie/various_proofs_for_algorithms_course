\documentclass[12pt]{article}
\usepackage{amsmath,amssymb,amsthm}
\begin{document}
\title{TCSS 343 - Challenge 1 - Proof by Induction}
\author{Jake McKenzie}
\maketitle
\[\sqrt{1+2\sqrt{1+3\sqrt{1+4\sqrt{1+5\sqrt{\dots}}}}} = 3\]\\
In order to properly prove this we will find a recurrance relationship. Infinite fractions and infinite roots of this nature are infact recurrance definitions. On their own, whenever a mathematician (or engineer/scientists) adds those dot dot dot to an equation it is meaningless. We need to be able to properly pin down just what it is those dots mean. \\\\
To start with, it is not immediately evident what doing something infinitely often means. It is easy to see, when broken down to component parts, that the expression above was generated recursively. \\\\
\textbf{Our definition for the recurrance relationship:}\\\\
\begin{align*}
P(0):3 &= \sqrt{1+2(4)}\\
P(1):4 &= \sqrt{1+3(5)}\\
P(2):5 &= \sqrt{1+4(6)}\\
P(3):6 &= \sqrt{1+5(7)}\\
&\vdots\\
P(n):n &= \sqrt{1+(n-1)(n+1)}\\
\end{align*}
We see above that The infinite radical is nothing but the continiuous application of a recurrance relationship. This recurrance relationship will be used with induction to show that the infinite radical is exactly equal to 3.\\\\
By taking this recurrance relation and writing some quick mathematica code applying the definition I obtain the immedate results with numerical precision of 25:
\begin{align*}
\intertext{2 iterations:} & 1.732050807568877293527446\\
\intertext{4 iterations:} & 2.559830165300117975151434\\
\intertext{8 iterations:} & 2.962723004279579609315694\\
\intertext{16 iterations:} & 2.999817917584576136433652\\
\intertext{32 iterations:} & 2.999999996651465098539593\\
\intertext{64 iterations:} & 2.999999999999999999081537\\
\intertext{128 iterations:} & 3.000000000000000000000000\\
\intertext{256 iterations:} & 3.000000000000000000000000\\
\end{align*}
The numbers from numerical analysis appear to be approaching 3, in fact with the numerical precision given Mathematica rounds this number to 3. Iteration of a recursively defined function is exactly the same as evaluation an infinite expression of this form. For this reason we will use the recursive definition to prove the expression. \\\\
\textbf{Base Case:}\\
\begin{align*}
P(1):\sqrt{1+2(4)} &= 3\\
\sqrt{1+8} &= 3\\
\sqrt{9} &= 3\\
\sqrt{3^2} &= 3\\
3 &= 3\\
\end{align*}
\textbf{Inductive Hypothesis:}\\
\begin{align*}
P(0):3 &= \sqrt{1+2(4)}\\
P(1):4 &= \sqrt{1+3(5)}\\
P(2):5 &= \sqrt{1+4(6)}\\
P(3):6 &= \sqrt{1+5(7)}\\
&\vdots\\
P(k):k &= \sqrt{1+(k-1)(k+1)}\\
\end{align*}\\\\\\\\\\\\\\\\
\textbf{Inductive Step:}\\
\begin{align*}
P(0):3 &= \sqrt{1+2(4)}\\
P(1):4 &= \sqrt{1+3(5)}\\
P(2):5 &= \sqrt{1+4(6)}\\
P(3):6 &= \sqrt{1+5(7)}\\
&\vdots\\
P(k):k &= \sqrt{1+(k-1)(k+1)}\\
P(k + 1):k + 1 &= \sqrt{1+k(k+2)}\\
\end{align*}\\
Dealing with radicals on their own with the given expression untenable, but this recurrance relation drastically reduces the complexity dramatically. Now we will introduce the inductive hypothesis.\\
\begin{align*}
k + 1 &= \sqrt{1+\sqrt{1+(k-1)(k+1)}(k+2)}\\
k + 1 &= \sqrt{1+\sqrt{1+k^2-1}(k+2)}\\
k + 1 &= \sqrt{1+\sqrt{k^2}(k+2)}\\
k + 1 &= \sqrt{1+k(k+2)}\\
k + 1 &= \sqrt{k^2+2k+1}\\
k + 1 &= \sqrt{(k+1)^2}\\
k + 1 &= k+1\\
&\hspace{0.3cm}\qedsymbol
\end{align*}
\end{document}