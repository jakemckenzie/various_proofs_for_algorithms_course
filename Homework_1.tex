
\documentclass[12pt]{article}
\usepackage{amsmath,amssymb,amsthm}
\begin{document}
\title{TCSS 343 - Assignment 1}
\maketitle


\section{2.1 Understand}

Prove the theorem below. 
Use a \textbf{direct proof} to find constants that satisfy the definition of \begin{math}\Theta(n^2)\end{math} or use the \textbf{limit test}.  
Make sure your proof is complete, concise, clear and precise.
\\\\\textbf{Theorem 1.} 
\begin{math}n^5-64n^3-n^2 \epsilon \Theta(n^5)\end{math}
\\\\For the first theorem a limit test was used to ensure that polynomial given was \begin{math}\Theta(n^5)\end{math}. For most limits of 
this nature it is best to factor the polynomial, this usually results in a short elegant solution. 
\[ \lim_{n \to \infty}  \frac{n^5-64n^3-n^2}{n^5} = \lim_{n \to \infty} \frac{n^5}{n^5}\Bigg(\frac{1-\frac{64}{n^2}-\frac{1}{n^3}}{1}\Bigg) = \lim_{n \to \infty} 1\Bigg(\frac{1-0-0}{1}\Bigg) = \lim_{n \to \infty} 1 = 1\]
\\Since the limit converged to 1, a positive constant value, we may say that theorem 1 is indeed true.\\\\


\noindent Prove Gauss’s sum using induction on \begin{math}n\end{math}. Make sure to include a base case for \begin{math} n = 1\end{math} and an inductive hypothesis and an inductive step for \begin{math}n > 1\end{math}.\\

\noindent \textbf{Theorem 2.}
\begin{math}P(n):\sum\limits_{i = 1}^{n} i = \frac{n(n + 1)}{2} \forall n \epsilon \mathbb{Z}_+  \end{math}\\
\\\\For this proof I used the fundamental theorem of algebra to help assist me. I hope that the steps are clear.\\
\begin{enumerate}
\item Base Case:\\
\[P(1):\sum\limits_{i = 1}^{1} i = 1  = \frac{1(1+1)}{2} = \frac{2}{2} = 1\]
\item Inductive Hypothesis:
\[P(k):\sum\limits_{i = 1}^{k} i = \frac{k(k + 1)}{2} \forall k \epsilon \mathbb{Z}_+\]
\item Inductive Step:

\begin{align*}
P(k+1):\sum\limits_{i = 1}^{k+1} i &= \frac{(k + 1)(k+2)}{2} \forall k \epsilon \mathbb{Z}_+\\
\sum\limits_{i = 1}^{k} i + (k + 1)&= \frac{(k + 1)(k+2)}{2}\\
\frac{k(k+1)}{2} + (k+1)&= \frac{(k + 1)(k+2)}{2}\\
\frac{k(k+1)}{2} + \frac{2(k+1)}{2}&= \frac{(k + 1)(k+2)}{2}\\
\frac{k(k+1)+ 2(k+1)}{2}&= \frac{(k + 1)(k+2)}{2}\\
\frac{(k + 1)(k+2)}{2}&= \frac{(k + 1)(k+2)}{2} \hspace{0.3cm}\qedsymbol
\end{align*}
\end{enumerate}
\noindent Prove Gauss’s sum using induction on \begin{math}n\end{math}. Make sure to include a base case for \begin{math} n = 1\end{math} and an inductive hypothesis and an inductive step for \begin{math}n > 1\end{math}.\\\\
\noindent \textbf{Theorem 3.}
\begin{math}P(n):\sum\limits_{i = 1}^{n} i^5 = \big(\frac{n(n + 1)}{2}\big)^2\frac{2n^2+2n-1}{3} \forall n \epsilon \mathbb{Z}_+  \end{math}
\\\\For this proof I used the fundamental theorem of algebra to help assist me. The last three steps of the inductive step to ensure \begin{math}P(k+1)\end{math} step I obtained those steps by expanding the right hand side and ensuring that each polynomial on the left hand side matched the right hand side. Those were included on the left hand side in reverse to ensure continuity.\\
\begin{enumerate}

\item Base Case:\\
\begin{align*}
P(1):\sum\limits_{i = 1}^{1} i^5 &= \big(\frac{1(1 + 1)}{2}\big)^2\frac{2(1)^2+2(1)-1}{3}\\
1^5 &= \big(\frac{1(2)}{2}\big)^2\frac{3}{3}\\
1 &= 1
\end{align*}

\item Inductive Hypothesis:
\begin{align*}
&P(k):\sum\limits_{i = 1}^{k} i^5 = \big(\frac{k(k + 1)}{2}\big)^2\frac{2k^2+2k-1}{3} \forall k \epsilon \mathbb{Z}_+
\end{align*}
\item Inductive Step:
\begin{align*}
P(k+1):\sum\limits_{i = 1}^{k+1} i^5 = \big(\frac{(k+1)(k + 1 + 1)}{2}\big)^2\frac{2(k+1)^2+2(k+1)-1}{3} \forall k \epsilon \mathbb{Z}_+ \\
\sum\limits_{i = 1}^{k} i^5 + (k + 1)^5 = \big(\frac{(k+1)(k + 2)}{2}\big)^2\frac{2k^2+5k+2}{3} \\
\bigg(\frac{k(k + 1)}{2}\bigg)^2\frac{2k^2+2k-1}{3} + (k + 1)^5 &=\\
\bigg(\frac{1}{12}\bigg)[k^2(k+1)^2(2k^2+2k-1)+12(k+1)^5]&= \\
\bigg(\frac{1}{12}\bigg)[(k^4+2k^3+k^2)(2k^2+2k-1)+12(k+1)^5]&=\\
\end{align*}
\begin{align*}
\bigg(\frac{1}{12}\bigg)(2k^6+6k^5+5k^4-k^2+12k^5+60k^4+120k^3+120k^2+60k+12))&=\\
\bigg(\frac{1}{12}\bigg)(2k^6+18k^5+65k^4+120k^3+119k^2+60k+12))&=\\
\bigg(\frac{1}{12}\bigg)(k^4+6k^3+13k^2+12k+4)(2k^2+6k+3)&=\\
\bigg(\frac{1}{12}\bigg)(k^2+3k+2)^2(2k^2+6k+3)&=\\
\big(\frac{(k+1)(k + 2)}{2}\big)^2\frac{2k^2+5k+2}{3}&=\hspace{0.3cm}\qedsymbol
\end{align*}
\end{enumerate}
Prove by induction that for all natural numbers \begin{math}x\end{math} and \begin{math}n\end{math}, \begin{math}x^n - 1\end{math} is divisible by \begin{math}x - 1\end{math}.\\\\
\noindent For this proof I decided to show that \begin{math}(x^n - 1)\bmod{x - 1}\equiv 0\end{math} because that statement is equivalent to saying \begin{math}x^n - 1\mid x - 1\end{math}.\\\\
\noindent Show \begin{math}\forall x,n \epsilon \mathbb{Z}_+ ;P(x,n):(x^n - 1)\bmod{x - 1}\equiv 0\end{math} 
\begin{enumerate}
\item Base Case:\\
\[P(1,1):(1^1 - 1)\bmod{1 - 1}\equiv 0 \mod 0 \equiv 0\]\\
This is assuming that we define \begin{math}a \mod 0 \equiv 0\end{math} where a may be any integer. This answer depends on how we decide to define the modulus function. This is assuming we define the modulus to be the distance of any two integers from each other. In many modern programming languages and applications this operation is undefined. I see no other way to complete this proof without ignoring this limitation and using the more liberal definition.
\item Inductive Hypothesis:
\[P(a,b):(a^b - 1)\bmod{a - 1}\equiv 0\ \forall a,b \epsilon \mathbb{Z}_+\]
\item Inductive Step:\\
For this proof I made heavy use of the binomial theorem, specifically the power expansion on \begin{math}(1+x)^n\end{math}.
\begin{align*}
P(a+1,b+1):((a+1)^{b+1} - 1)\bmod{(a + 1 - 1)}&\equiv 0\ ; \forall a,b \epsilon \mathbb{Z}_+\\
((a+1)(a+1)^b - 1)\bmod{a}&\equiv 0\\
((a+1)\sum\limits_{i = 0}^{b} {b \choose i} a^i - 1)\bmod{a}&\equiv 0\\
(\sum\limits_{i = 0}^{b} {b \choose i} a^{i+1} + \sum\limits_{i = 0}^{b} {b \choose i} a^i - 1)\bmod{a}&\equiv 0\\
(\sum\limits_{i = 0}^{b} {b \choose i} a^{i+1} + \sum\limits_{i = 1}^{b - 1} {b \choose i} a^i + {b \choose b}(a^b + 1) - 1)\bmod{a}&\equiv 0\\
(\sum\limits_{i = 0}^{b} {b \choose i} a^{i+1} + \sum\limits_{i = 1}^{b - 1} {b \choose i} a^i + (1)(a^b + 1 - 1)\bmod{a}&\equiv 0\\
(\sum\limits_{i = 0}^{b} {b \choose i} a^{i+1} + \sum\limits_{i = 1}^{b - 1} {b \choose i} a^i + a^b)\bmod{a}&\equiv 0\\
((\sum\limits_{i = 0}^{b} {b \choose i} a^{i+1})\bmod{a} + (\sum\limits_{i = 1}^{b - 1} {b \choose i} a^i)\bmod{a} + (a^b)\bmod{a})\bmod{a}&\equiv 0\\
(0 + 0 + 0)\bmod{a}&\equiv 0\\
0\bmod{a}&\equiv 0\\
0&\equiv 0\\
\end{align*}
\end{enumerate}
\end{document}