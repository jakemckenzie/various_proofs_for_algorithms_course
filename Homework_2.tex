\documentclass[paper=a4,fontsize=11pt]{article}
\usepackage{amsmath,amssymb,amsthm}
\usepackage[protrusion=true,expansion=true]{microtype}	
\usepackage{algorithm}
\usepackage{algpseudocode}
\usepackage[margin=1.5in]{geometry}


\begin{document}
\title{TCSS 343 - Assignment 1}
\author{Jake McKenzie}
\maketitle
\begin{enumerate}
\item [(3 points) 1.] Below is a self-reduction for the \texttt{MAX} problem. State a recursive algorithm using pseudocode for finding the maximum element based on this self-reduction.\\
\begin{algorithm}
\caption{Find Max integer in an Array with simple recursion}
\label{array-sum}
\begin{algorithmic}[1]
\Procedure{Find Max}{$A,a$}
    \If{($a == b$)}
        \State return A[a]
    \ElsIf{(a $<$ b)}
        \State Max(A[a],Find Max(a + 1))
    \EndIf
\EndProcedure
\Procedure{Max}{$a,b$}
    return (a $<$ b) ? b : a
\EndProcedure
\end{algorithmic}
\end{algorithm}
\item [(6 points) 2.] Using the same reduction as part 1 now state a recurrence $T(n)$ that expresses the worst case run time of the recursive algorithm. Find a similar recurrence in your notes and state the tight bound on $T(n)$.\\\\\
Line 3 makes $1$ amount of operations while line 5 makes $T(n-1)$, this is because there are $n-1$ amount of comparisons to check for the max in the recurrence for when this list is greater than 1. **Note**: Consistency of whether the constant amount of operations is writen as $1$ or $O(1)$ are inconsistent so I went with the notation I've seen the most used often.
\[
  T(n) =
    \begin{cases}
        1 & \text{if $n = 1$} \\
        T(n-1) + 1 & \text{if $n > 1$}
    \end{cases}
\]
Claim: $\forall n > 0$, the running time of \textit{Find Max} $\epsilon O(n)$. We consder the recurrance relation above.
\begin{enumerate}
\item[1.] Base Case:\\
$$n = 1; T(1) = 1$$
\item[2.] Inductive Hypothesis:\\
\[
  T(k) =
    \begin{cases}
        1 & \text{if $k = 1$} \\
        T(k-1) + 1 & \text{if $k > 1$}
    \end{cases}
\]
\\Assume for an arbitrary $k, T(k) \leq k$
\item[3.] Inductive Step:
\begin{gather*}
\text{if $k + 1 > 1$}\\
T(k + 1) = T(k) + 1\\
T(k + 1) = k + 1 + 1\\  
T(k + 1) = k + 2\\
T(k + 1) \epsilon O(k)
\end{gather*}
\end{enumerate}
\item [(9 points) 3.] Below is a self-reduction for the \texttt{MAX} problem. State a recursive algorithm using pseudocode for finding the maximum element based on this self-reduction.\\\\\
\[
M(A[a\dots b]) = \left\{
\begin{array}{cl}
-\infty & \textrm{ if } a > b\\
A[a] & \textrm{ if } a = b\\
\max(M(A[a\dots t_1]), \max(M(A[t_1+1\dots t_2]), M(A[t_2+1\dots b]))) & \textrm{ if } a < b
\end{array}
\right.
\]

\end{enumerate}
\end{document}